\documentclass[english]{tktltiki}
\usepackage[pdftex]{graphicx}
\usepackage{graphicx}
\usepackage{subfigure}
\usepackage{url}
\usepackage{xcolor}
\usepackage{amsmath}
\begin{document}


\onehalfspacing
\title{Reinforcement Learning in Games: A Case Study}
\author{Ege Can �zer}
\date{\today}

\maketitle
\numberofpagesinformation{\numberofpages\ pages + \numberofappendixpages\ appendices}

\classification{\protect{\ \\
	A.1 [Introductory and Survey],\\
	I.7.m [Document and text processing]}}

\keywords{Reinforcement Learning}

\begin{abstract}
    Self-learning programs have been studied and applied in many different fields; but recently, it gains more popularity and familiarity due to its breakthrough success in the game domain. Unlike the examples are taken from the real world, having the pre-defined set of rules and the less complicated environment in this domain provides greater flexibility to develop reinforcement learning algorithms. In this paper, we will study the applications of reinforcement learning algorithms in the game context by closely focusing on five different articles. Based on the findings, we hope to propose possible improvements for future studies.
\end{abstract}

\mytableofcontents

\section{Introduction}
During the last decade, reinforcement learning paradigm, despite being actually not a new concept, has gained decent amount of popularity. Starting from middle 80's and ever since then, it has been applied in many fields to address complex problems such as in robotics, control systems, finance, and agent-based systems due to its generic formulation to any problems. In its essence, reinforcement learning concept looks for optimal input actions that maximizes the output states by making use of the feedbacks from the environment. Inspite of the fact that RL being a mature and widely-used concept, the primary reason to get such attention recently is due to breakthrough success in the game context.

Studying reinforcement learning algorithms in the game context contributes several definitive advantages. In general, modelling of the problem, implementation, and analysis of the results in real life examples are harder than the artificial ones such as games. Moreover, having simplified and controlled environment, as well as specificly defined rule sets not only allow an oppurtinity to explore the new variety of RL algorithms, but also enable have concrete evaluation measurements. Further, research results are reproducable, easy to simulate and provide test-bed for future studies. Therefore, for the given reasons, studying reinforcement learning in games can help the field to progress more efficiently.

different modelling paradigms can contribute to another aspects of the other research fields for return, which makes RL in games worthwhile to study. In this paper, we will take a look at different systems that has different modelling assumptions as well as different RL algorithms. 

In this paper, we review five different reinforcement learning example systems, where each has distinct approaches and modelling assumptions to solve the problem they are addressed to. The rest of the paper is as follows. The following section present brief bacground information regarding those articles. In section 2, we examine each articles in detail, how do they tackle the problem, and how RL is used to resove these. In section 3, we propose possible improvements that can be taken to help to progress the study field. Last section, summarizes and concludes the paper.

\section{Application of Reinforcement Learning to Games}
\subsection{General outline for articles}
\begin{itemize}
    \item What is the problem, introduce environment
    \item Proposed solution to the problem
    \item Results
\end{itemize}
\subsection{Mario: Q-Learning}
Everything is defined in this model. Low-level design of the environment, everything is known.
\subsection{Civilization 4: High Level RL Approach}
RL used as a high level strategy planning, they let the game's AI handle the low-level operations.
\subsection{TD Gammon: First Success}
First important step towards AI in games using RL.
\subsection{Atari Games: Context Free Approach DQN}
Raw sensory input data to train bunch of different games without knowing its internal elements explicitly. I need to read this more deeply.
\subsection{Mastering the Game of Go}
Read this again!

\section{Future Researches}
\begin{itemize}
    \item Evaluation frameworks, point out many systems, human comparison
    \item Sandbox learning, can this be used to train DQN approach and applied on other games? This way one may not to deal with the context at all? RL in games, may help other 
    \item Need for a theoratical framework to describe the RL papers that encapsulating some context. Similar frameworks exist in the view of computational creativity world. This would provide more structural way for reseaches to present their case studies, but also to describe the candidate systems regardless of what they use. A theoretical framework, such as that presented here, may be useful in teasing out philosophical issues, but it may also be useful in giving generalised descriptions of behaviours which might be observed in creative agents. Self-critics about writing this paper, more structured way to review RL systems in terms of some theoratical framework.
\end{itemize}

\section{Conclusion}
Recap everything

\nocite{*}
\bibliographystyle{tktl}
\bibliography{lahteet}

\lastpage

\appendices

\pagestyle{empty}



\end{document}


